\documentclass{beamer}

% ************* Preamble *************************** 
% Language setting
% Replace `english' with e.g. `spanish' to change the document language
\usepackage[english]{babel}

% Useful packages
\usepackage{amsmath}
% \usepackage{graphicx} %already loaded by beamer document class
\graphicspath{{../figures/}}
% \usepackage{hyperref} %already loaded by beamer document class
\usepackage{blindtext} % to generate 2 pages of text
\usepackage{booktabs} % nicer tables w/ mid, top, bottom rules
\usepackage{dcolumn} % should align decimals in tables
\newcolumntype{d}[1]{D{.}{.}{#1}} % command for dcolumn, use d{2} instead of l,
% or c in tabular column specifier

 

% Dont use caption/ subcaption in beamer, use columns instead.
\usepackage[compatibility=false]{caption} % for subfigures (panels)
\usepackage{subcaption} % for subfigures (panels)
% \setbeamertemplate{caption}[numbered] %to get figures to be numbered in beamer class

% Biblatex (options taken from Igors Github, is it equal to JoF style?)
\usepackage[
    backend=biber,
    style=bwl-FU,
    url=false,
    doi=false,
    eprint=false
]{biblatex}
% \addbibresource{references.bib}


% Title
\title{Presentation title}

% \author[test]{Egemen Erdogdu\thanks{Egemen Erdogdu, University of Zurich, Rämistrasse 71, 8006 Zürich} \and Jonas Schmidiger\thanks{Jonas Schmidiger, University of Zurich, Rämistrasse 71, 8006 Zürich} \and Mathias Ruoss\thanks{Mathias Ruoss, University of Zurich, Rämistrasse 71, 8006 Zürich}}
\author[E. Erdogdu, J. Schmidiger, M. Ruoss] % (optional, for multiple authors)
{E. Erdogdu\inst{1} \and J. Schmidiger\inst{2} \and M. Ruoss\inst{3}}

\institute[UZH] % (optional)
{
  \inst{1}%
  Faculty of Banking and Finance\\
  Very Famous University
  \and
  \inst{2}%
  Faculty of Banking and Finance\\
  Very Famous University
  \and
  \inst{3}%
  Faculty of Banking and Finance\\
  Very Famous University
}
\date{\today}


% \usetheme[]{Warsaw}
% \usecolortheme{orchid}

% ***********************************************
\begin{document}

\frame{\titlepage}

% % Table of Contents
% \begin{frame}
%     \frametitle{Table of Contents}
%     \tableofcontents    
% \end{frame}

% % \section{Section 1}
% \begin{frame}
%     \frametitle{A theorem}
%     \framesubtitle{Frame subtitle}
%     \begin{theorem}
%         $a^2 + b^2 = c^2$
%     \end{theorem}
% \end{frame}

% % \section{Section 2}
% \begin{frame}
%     \frametitle{Some equations}
%     \framesubtitle{Frame subtitle}
%     % \begin{equation}
%         \begin{align}
%             y + x = 3\\
%             y + 3 + 4 = 10\\
%             y + \mathbf{x} + Z = 3
%         \end{align}
%     % \end{equation}
% \end{frame}

% % \section{Section 3}
% \begin{frame}
%     \frametitle{A figure with two panels}
%     \framesubtitle{A Subtitle}
%     \begin{figure}
%     \centering
%       \begin{subfigure}{0.45\textwidth}
%         \centering
%         \includegraphics[width=\textwidth]{stock-index-weird-plot}
%         \caption{First subfigure}
%         \label{fig:a}
%       \end{subfigure}\hfill
%       \begin{subfigure}{0.45\textwidth}
%         \centering
%         % Figures taken from igors github
%         \includegraphics[width=\textwidth]{beta-vs-mu}
%         \caption{Second subfigure}
%         \label{fig:b}
%       \end{subfigure}\\
%     \caption{Two panels side by side}\label{fig:stock-beta-comparison}
%     \end{figure}
% \end{frame}


% \begin{frame}
%     \frametitle{Another figure with two panels aligned}
%     \framesubtitle{Another subtitle}
%     \begin{figure}[H]
%         \centering
%         \subcaptionbox{First subfigure}%
%           [.45\linewidth]{\includegraphics[height=3.5cm]{stock-index-weird-plot.png}}
%         \subcaptionbox{Second subfigure}
%           [.45\linewidth]{\includegraphics[height=2cm]{beta-vs-mu.png}}
%     \caption{Two panels side by side}    
%     \end{figure}        
% \end{frame}


% Exercise 1 of lecture 6-Visualization
\begin{frame}
  \frametitle{A table with the package \texttt{dcolumn}}
\begin{table}[]
  \resizebox{\textwidth}{!}{
  \begin{tabular}{@{}ld{3}d{3}d{3}d{3}d{3}d{3}@{}}
  \toprule
  \multicolumn{1}{l}{Stock}       & \multicolumn{1}{c}{FB} & \multicolumn{1}{c}{AMZN} & \multicolumn{1}{c}{AAPL} & \multicolumn{1}{c}{NFLX} & \multicolumn{1}{c}{GOOGL} &  \\ \midrule
  Panel A: Descriptive statistics &        &        &        &        &        &  \\ \midrule
  Mean                            & 0.083  & 0.065  & 0.112  & 0.051  & 0.091  &  \\
  Variance                        & 0.041  & 0.030  & 0.044  & 0.061  & 0.037  &  \\
  Median                          & 0.064  & 0.058  & 0.091  & 0.043  & 0.102  &  \\
  Maximum                         & 0.053  & 0.064  & 0.078  & 0.043  & 0.062  &  \\
  Minimum                         & -0.211 & -0.181 & -0.195 & -0.203 & -0.081 &  \\
  Jarque Bera                     & 95.35   & 81.54   & 49.27   & 40.58   & 0.58   &  \\ \bottomrule
  \end{tabular}
  }
  \end{table}
\end{frame}


% Exercise 2, table as heatmap
\begin{frame}
    \frametitle{A table as a heatmap}
    % \framesubtitle{Another subtitle}
    \begin{figure}[H]
      \centering
      \includegraphics[width=\linewidth]{heatmap.png}
      % \caption{heatmap of returns}    
    \end{figure}        
\end{frame}


% Exercise 3, line plot with color friendly palette -> see code in "lineplot.py"
% we used the option palette="colorblind" in seaborn, a python package
\begin{frame}
  \frametitle{A table as a lineplot}
  % \framesubtitle{Another subtitle}
  \begin{figure}[H]
    \centering
    \includegraphics[width=\linewidth]{lineplot.png}
    % \caption{heatmap of returns}    
  \end{figure}        
\end{frame}

% Answer exercise 4:
% Grids could be pre-attentive, as the human brain learns spatial information
% from it. It also could be a learned pattern, e.g. through our school system, as we 
% learn how to read and write on sheets with horizontal and vertical grids.

\end{document}